
% Default to the notebook output style

    


% Inherit from the specified cell style.




    
\documentclass[11pt]{article}

    
    
    \usepackage[T1]{fontenc}
    % Nicer default font (+ math font) than Computer Modern for most use cases
    \usepackage{mathpazo}

    % Basic figure setup, for now with no caption control since it's done
    % automatically by Pandoc (which extracts ![](path) syntax from Markdown).
    \usepackage{graphicx}
    % We will generate all images so they have a width \maxwidth. This means
    % that they will get their normal width if they fit onto the page, but
    % are scaled down if they would overflow the margins.
    \makeatletter
    \def\maxwidth{\ifdim\Gin@nat@width>\linewidth\linewidth
    \else\Gin@nat@width\fi}
    \makeatother
    \let\Oldincludegraphics\includegraphics
    % Set max figure width to be 80% of text width, for now hardcoded.
    \renewcommand{\includegraphics}[1]{\Oldincludegraphics[width=.8\maxwidth]{#1}}
    % Ensure that by default, figures have no caption (until we provide a
    % proper Figure object with a Caption API and a way to capture that
    % in the conversion process - todo).
    \usepackage{caption}
    \DeclareCaptionLabelFormat{nolabel}{}
    \captionsetup{labelformat=nolabel}

    \usepackage{adjustbox} % Used to constrain images to a maximum size 
    \usepackage{xcolor} % Allow colors to be defined
    \usepackage{enumerate} % Needed for markdown enumerations to work
    \usepackage{geometry} % Used to adjust the document margins
    \usepackage{amsmath} % Equations
    \usepackage{amssymb} % Equations
    \usepackage{textcomp} % defines textquotesingle
    % Hack from http://tex.stackexchange.com/a/47451/13684:
    \AtBeginDocument{%
        \def\PYZsq{\textquotesingle}% Upright quotes in Pygmentized code
    }
    \usepackage{upquote} % Upright quotes for verbatim code
    \usepackage{eurosym} % defines \euro
    \usepackage[mathletters]{ucs} % Extended unicode (utf-8) support
    \usepackage[utf8x]{inputenc} % Allow utf-8 characters in the tex document
    \usepackage{fancyvrb} % verbatim replacement that allows latex
    \usepackage{grffile} % extends the file name processing of package graphics 
                         % to support a larger range 
    % The hyperref package gives us a pdf with properly built
    % internal navigation ('pdf bookmarks' for the table of contents,
    % internal cross-reference links, web links for URLs, etc.)
    \usepackage{hyperref}
    \usepackage{longtable} % longtable support required by pandoc >1.10
    \usepackage{booktabs}  % table support for pandoc > 1.12.2
    \usepackage[inline]{enumitem} % IRkernel/repr support (it uses the enumerate* environment)
    \usepackage[normalem]{ulem} % ulem is needed to support strikethroughs (\sout)
                                % normalem makes italics be italics, not underlines
    

    
    
    % Colors for the hyperref package
    \definecolor{urlcolor}{rgb}{0,.145,.698}
    \definecolor{linkcolor}{rgb}{.71,0.21,0.01}
    \definecolor{citecolor}{rgb}{.12,.54,.11}

    % ANSI colors
    \definecolor{ansi-black}{HTML}{3E424D}
    \definecolor{ansi-black-intense}{HTML}{282C36}
    \definecolor{ansi-red}{HTML}{E75C58}
    \definecolor{ansi-red-intense}{HTML}{B22B31}
    \definecolor{ansi-green}{HTML}{00A250}
    \definecolor{ansi-green-intense}{HTML}{007427}
    \definecolor{ansi-yellow}{HTML}{DDB62B}
    \definecolor{ansi-yellow-intense}{HTML}{B27D12}
    \definecolor{ansi-blue}{HTML}{208FFB}
    \definecolor{ansi-blue-intense}{HTML}{0065CA}
    \definecolor{ansi-magenta}{HTML}{D160C4}
    \definecolor{ansi-magenta-intense}{HTML}{A03196}
    \definecolor{ansi-cyan}{HTML}{60C6C8}
    \definecolor{ansi-cyan-intense}{HTML}{258F8F}
    \definecolor{ansi-white}{HTML}{C5C1B4}
    \definecolor{ansi-white-intense}{HTML}{A1A6B2}

    % commands and environments needed by pandoc snippets
    % extracted from the output of `pandoc -s`
    \providecommand{\tightlist}{%
      \setlength{\itemsep}{0pt}\setlength{\parskip}{0pt}}
    \DefineVerbatimEnvironment{Highlighting}{Verbatim}{commandchars=\\\{\}}
    % Add ',fontsize=\small' for more characters per line
    \newenvironment{Shaded}{}{}
    \newcommand{\KeywordTok}[1]{\textcolor[rgb]{0.00,0.44,0.13}{\textbf{{#1}}}}
    \newcommand{\DataTypeTok}[1]{\textcolor[rgb]{0.56,0.13,0.00}{{#1}}}
    \newcommand{\DecValTok}[1]{\textcolor[rgb]{0.25,0.63,0.44}{{#1}}}
    \newcommand{\BaseNTok}[1]{\textcolor[rgb]{0.25,0.63,0.44}{{#1}}}
    \newcommand{\FloatTok}[1]{\textcolor[rgb]{0.25,0.63,0.44}{{#1}}}
    \newcommand{\CharTok}[1]{\textcolor[rgb]{0.25,0.44,0.63}{{#1}}}
    \newcommand{\StringTok}[1]{\textcolor[rgb]{0.25,0.44,0.63}{{#1}}}
    \newcommand{\CommentTok}[1]{\textcolor[rgb]{0.38,0.63,0.69}{\textit{{#1}}}}
    \newcommand{\OtherTok}[1]{\textcolor[rgb]{0.00,0.44,0.13}{{#1}}}
    \newcommand{\AlertTok}[1]{\textcolor[rgb]{1.00,0.00,0.00}{\textbf{{#1}}}}
    \newcommand{\FunctionTok}[1]{\textcolor[rgb]{0.02,0.16,0.49}{{#1}}}
    \newcommand{\RegionMarkerTok}[1]{{#1}}
    \newcommand{\ErrorTok}[1]{\textcolor[rgb]{1.00,0.00,0.00}{\textbf{{#1}}}}
    \newcommand{\NormalTok}[1]{{#1}}
    
    % Additional commands for more recent versions of Pandoc
    \newcommand{\ConstantTok}[1]{\textcolor[rgb]{0.53,0.00,0.00}{{#1}}}
    \newcommand{\SpecialCharTok}[1]{\textcolor[rgb]{0.25,0.44,0.63}{{#1}}}
    \newcommand{\VerbatimStringTok}[1]{\textcolor[rgb]{0.25,0.44,0.63}{{#1}}}
    \newcommand{\SpecialStringTok}[1]{\textcolor[rgb]{0.73,0.40,0.53}{{#1}}}
    \newcommand{\ImportTok}[1]{{#1}}
    \newcommand{\DocumentationTok}[1]{\textcolor[rgb]{0.73,0.13,0.13}{\textit{{#1}}}}
    \newcommand{\AnnotationTok}[1]{\textcolor[rgb]{0.38,0.63,0.69}{\textbf{\textit{{#1}}}}}
    \newcommand{\CommentVarTok}[1]{\textcolor[rgb]{0.38,0.63,0.69}{\textbf{\textit{{#1}}}}}
    \newcommand{\VariableTok}[1]{\textcolor[rgb]{0.10,0.09,0.49}{{#1}}}
    \newcommand{\ControlFlowTok}[1]{\textcolor[rgb]{0.00,0.44,0.13}{\textbf{{#1}}}}
    \newcommand{\OperatorTok}[1]{\textcolor[rgb]{0.40,0.40,0.40}{{#1}}}
    \newcommand{\BuiltInTok}[1]{{#1}}
    \newcommand{\ExtensionTok}[1]{{#1}}
    \newcommand{\PreprocessorTok}[1]{\textcolor[rgb]{0.74,0.48,0.00}{{#1}}}
    \newcommand{\AttributeTok}[1]{\textcolor[rgb]{0.49,0.56,0.16}{{#1}}}
    \newcommand{\InformationTok}[1]{\textcolor[rgb]{0.38,0.63,0.69}{\textbf{\textit{{#1}}}}}
    \newcommand{\WarningTok}[1]{\textcolor[rgb]{0.38,0.63,0.69}{\textbf{\textit{{#1}}}}}
    
    
    % Define a nice break command that doesn't care if a line doesn't already
    % exist.
    \def\br{\hspace*{\fill} \\* }
    % Math Jax compatability definitions
    \def\gt{>}
    \def\lt{<}
    % Document parameters
    \title{Statistics-from-Stock-Data}
    
    
    

    % Pygments definitions
    
\makeatletter
\def\PY@reset{\let\PY@it=\relax \let\PY@bf=\relax%
    \let\PY@ul=\relax \let\PY@tc=\relax%
    \let\PY@bc=\relax \let\PY@ff=\relax}
\def\PY@tok#1{\csname PY@tok@#1\endcsname}
\def\PY@toks#1+{\ifx\relax#1\empty\else%
    \PY@tok{#1}\expandafter\PY@toks\fi}
\def\PY@do#1{\PY@bc{\PY@tc{\PY@ul{%
    \PY@it{\PY@bf{\PY@ff{#1}}}}}}}
\def\PY#1#2{\PY@reset\PY@toks#1+\relax+\PY@do{#2}}

\expandafter\def\csname PY@tok@w\endcsname{\def\PY@tc##1{\textcolor[rgb]{0.73,0.73,0.73}{##1}}}
\expandafter\def\csname PY@tok@c\endcsname{\let\PY@it=\textit\def\PY@tc##1{\textcolor[rgb]{0.25,0.50,0.50}{##1}}}
\expandafter\def\csname PY@tok@cp\endcsname{\def\PY@tc##1{\textcolor[rgb]{0.74,0.48,0.00}{##1}}}
\expandafter\def\csname PY@tok@k\endcsname{\let\PY@bf=\textbf\def\PY@tc##1{\textcolor[rgb]{0.00,0.50,0.00}{##1}}}
\expandafter\def\csname PY@tok@kp\endcsname{\def\PY@tc##1{\textcolor[rgb]{0.00,0.50,0.00}{##1}}}
\expandafter\def\csname PY@tok@kt\endcsname{\def\PY@tc##1{\textcolor[rgb]{0.69,0.00,0.25}{##1}}}
\expandafter\def\csname PY@tok@o\endcsname{\def\PY@tc##1{\textcolor[rgb]{0.40,0.40,0.40}{##1}}}
\expandafter\def\csname PY@tok@ow\endcsname{\let\PY@bf=\textbf\def\PY@tc##1{\textcolor[rgb]{0.67,0.13,1.00}{##1}}}
\expandafter\def\csname PY@tok@nb\endcsname{\def\PY@tc##1{\textcolor[rgb]{0.00,0.50,0.00}{##1}}}
\expandafter\def\csname PY@tok@nf\endcsname{\def\PY@tc##1{\textcolor[rgb]{0.00,0.00,1.00}{##1}}}
\expandafter\def\csname PY@tok@nc\endcsname{\let\PY@bf=\textbf\def\PY@tc##1{\textcolor[rgb]{0.00,0.00,1.00}{##1}}}
\expandafter\def\csname PY@tok@nn\endcsname{\let\PY@bf=\textbf\def\PY@tc##1{\textcolor[rgb]{0.00,0.00,1.00}{##1}}}
\expandafter\def\csname PY@tok@ne\endcsname{\let\PY@bf=\textbf\def\PY@tc##1{\textcolor[rgb]{0.82,0.25,0.23}{##1}}}
\expandafter\def\csname PY@tok@nv\endcsname{\def\PY@tc##1{\textcolor[rgb]{0.10,0.09,0.49}{##1}}}
\expandafter\def\csname PY@tok@no\endcsname{\def\PY@tc##1{\textcolor[rgb]{0.53,0.00,0.00}{##1}}}
\expandafter\def\csname PY@tok@nl\endcsname{\def\PY@tc##1{\textcolor[rgb]{0.63,0.63,0.00}{##1}}}
\expandafter\def\csname PY@tok@ni\endcsname{\let\PY@bf=\textbf\def\PY@tc##1{\textcolor[rgb]{0.60,0.60,0.60}{##1}}}
\expandafter\def\csname PY@tok@na\endcsname{\def\PY@tc##1{\textcolor[rgb]{0.49,0.56,0.16}{##1}}}
\expandafter\def\csname PY@tok@nt\endcsname{\let\PY@bf=\textbf\def\PY@tc##1{\textcolor[rgb]{0.00,0.50,0.00}{##1}}}
\expandafter\def\csname PY@tok@nd\endcsname{\def\PY@tc##1{\textcolor[rgb]{0.67,0.13,1.00}{##1}}}
\expandafter\def\csname PY@tok@s\endcsname{\def\PY@tc##1{\textcolor[rgb]{0.73,0.13,0.13}{##1}}}
\expandafter\def\csname PY@tok@sd\endcsname{\let\PY@it=\textit\def\PY@tc##1{\textcolor[rgb]{0.73,0.13,0.13}{##1}}}
\expandafter\def\csname PY@tok@si\endcsname{\let\PY@bf=\textbf\def\PY@tc##1{\textcolor[rgb]{0.73,0.40,0.53}{##1}}}
\expandafter\def\csname PY@tok@se\endcsname{\let\PY@bf=\textbf\def\PY@tc##1{\textcolor[rgb]{0.73,0.40,0.13}{##1}}}
\expandafter\def\csname PY@tok@sr\endcsname{\def\PY@tc##1{\textcolor[rgb]{0.73,0.40,0.53}{##1}}}
\expandafter\def\csname PY@tok@ss\endcsname{\def\PY@tc##1{\textcolor[rgb]{0.10,0.09,0.49}{##1}}}
\expandafter\def\csname PY@tok@sx\endcsname{\def\PY@tc##1{\textcolor[rgb]{0.00,0.50,0.00}{##1}}}
\expandafter\def\csname PY@tok@m\endcsname{\def\PY@tc##1{\textcolor[rgb]{0.40,0.40,0.40}{##1}}}
\expandafter\def\csname PY@tok@gh\endcsname{\let\PY@bf=\textbf\def\PY@tc##1{\textcolor[rgb]{0.00,0.00,0.50}{##1}}}
\expandafter\def\csname PY@tok@gu\endcsname{\let\PY@bf=\textbf\def\PY@tc##1{\textcolor[rgb]{0.50,0.00,0.50}{##1}}}
\expandafter\def\csname PY@tok@gd\endcsname{\def\PY@tc##1{\textcolor[rgb]{0.63,0.00,0.00}{##1}}}
\expandafter\def\csname PY@tok@gi\endcsname{\def\PY@tc##1{\textcolor[rgb]{0.00,0.63,0.00}{##1}}}
\expandafter\def\csname PY@tok@gr\endcsname{\def\PY@tc##1{\textcolor[rgb]{1.00,0.00,0.00}{##1}}}
\expandafter\def\csname PY@tok@ge\endcsname{\let\PY@it=\textit}
\expandafter\def\csname PY@tok@gs\endcsname{\let\PY@bf=\textbf}
\expandafter\def\csname PY@tok@gp\endcsname{\let\PY@bf=\textbf\def\PY@tc##1{\textcolor[rgb]{0.00,0.00,0.50}{##1}}}
\expandafter\def\csname PY@tok@go\endcsname{\def\PY@tc##1{\textcolor[rgb]{0.53,0.53,0.53}{##1}}}
\expandafter\def\csname PY@tok@gt\endcsname{\def\PY@tc##1{\textcolor[rgb]{0.00,0.27,0.87}{##1}}}
\expandafter\def\csname PY@tok@err\endcsname{\def\PY@bc##1{\setlength{\fboxsep}{0pt}\fcolorbox[rgb]{1.00,0.00,0.00}{1,1,1}{\strut ##1}}}
\expandafter\def\csname PY@tok@kc\endcsname{\let\PY@bf=\textbf\def\PY@tc##1{\textcolor[rgb]{0.00,0.50,0.00}{##1}}}
\expandafter\def\csname PY@tok@kd\endcsname{\let\PY@bf=\textbf\def\PY@tc##1{\textcolor[rgb]{0.00,0.50,0.00}{##1}}}
\expandafter\def\csname PY@tok@kn\endcsname{\let\PY@bf=\textbf\def\PY@tc##1{\textcolor[rgb]{0.00,0.50,0.00}{##1}}}
\expandafter\def\csname PY@tok@kr\endcsname{\let\PY@bf=\textbf\def\PY@tc##1{\textcolor[rgb]{0.00,0.50,0.00}{##1}}}
\expandafter\def\csname PY@tok@bp\endcsname{\def\PY@tc##1{\textcolor[rgb]{0.00,0.50,0.00}{##1}}}
\expandafter\def\csname PY@tok@fm\endcsname{\def\PY@tc##1{\textcolor[rgb]{0.00,0.00,1.00}{##1}}}
\expandafter\def\csname PY@tok@vc\endcsname{\def\PY@tc##1{\textcolor[rgb]{0.10,0.09,0.49}{##1}}}
\expandafter\def\csname PY@tok@vg\endcsname{\def\PY@tc##1{\textcolor[rgb]{0.10,0.09,0.49}{##1}}}
\expandafter\def\csname PY@tok@vi\endcsname{\def\PY@tc##1{\textcolor[rgb]{0.10,0.09,0.49}{##1}}}
\expandafter\def\csname PY@tok@vm\endcsname{\def\PY@tc##1{\textcolor[rgb]{0.10,0.09,0.49}{##1}}}
\expandafter\def\csname PY@tok@sa\endcsname{\def\PY@tc##1{\textcolor[rgb]{0.73,0.13,0.13}{##1}}}
\expandafter\def\csname PY@tok@sb\endcsname{\def\PY@tc##1{\textcolor[rgb]{0.73,0.13,0.13}{##1}}}
\expandafter\def\csname PY@tok@sc\endcsname{\def\PY@tc##1{\textcolor[rgb]{0.73,0.13,0.13}{##1}}}
\expandafter\def\csname PY@tok@dl\endcsname{\def\PY@tc##1{\textcolor[rgb]{0.73,0.13,0.13}{##1}}}
\expandafter\def\csname PY@tok@s2\endcsname{\def\PY@tc##1{\textcolor[rgb]{0.73,0.13,0.13}{##1}}}
\expandafter\def\csname PY@tok@sh\endcsname{\def\PY@tc##1{\textcolor[rgb]{0.73,0.13,0.13}{##1}}}
\expandafter\def\csname PY@tok@s1\endcsname{\def\PY@tc##1{\textcolor[rgb]{0.73,0.13,0.13}{##1}}}
\expandafter\def\csname PY@tok@mb\endcsname{\def\PY@tc##1{\textcolor[rgb]{0.40,0.40,0.40}{##1}}}
\expandafter\def\csname PY@tok@mf\endcsname{\def\PY@tc##1{\textcolor[rgb]{0.40,0.40,0.40}{##1}}}
\expandafter\def\csname PY@tok@mh\endcsname{\def\PY@tc##1{\textcolor[rgb]{0.40,0.40,0.40}{##1}}}
\expandafter\def\csname PY@tok@mi\endcsname{\def\PY@tc##1{\textcolor[rgb]{0.40,0.40,0.40}{##1}}}
\expandafter\def\csname PY@tok@il\endcsname{\def\PY@tc##1{\textcolor[rgb]{0.40,0.40,0.40}{##1}}}
\expandafter\def\csname PY@tok@mo\endcsname{\def\PY@tc##1{\textcolor[rgb]{0.40,0.40,0.40}{##1}}}
\expandafter\def\csname PY@tok@ch\endcsname{\let\PY@it=\textit\def\PY@tc##1{\textcolor[rgb]{0.25,0.50,0.50}{##1}}}
\expandafter\def\csname PY@tok@cm\endcsname{\let\PY@it=\textit\def\PY@tc##1{\textcolor[rgb]{0.25,0.50,0.50}{##1}}}
\expandafter\def\csname PY@tok@cpf\endcsname{\let\PY@it=\textit\def\PY@tc##1{\textcolor[rgb]{0.25,0.50,0.50}{##1}}}
\expandafter\def\csname PY@tok@c1\endcsname{\let\PY@it=\textit\def\PY@tc##1{\textcolor[rgb]{0.25,0.50,0.50}{##1}}}
\expandafter\def\csname PY@tok@cs\endcsname{\let\PY@it=\textit\def\PY@tc##1{\textcolor[rgb]{0.25,0.50,0.50}{##1}}}

\def\PYZbs{\char`\\}
\def\PYZus{\char`\_}
\def\PYZob{\char`\{}
\def\PYZcb{\char`\}}
\def\PYZca{\char`\^}
\def\PYZam{\char`\&}
\def\PYZlt{\char`\<}
\def\PYZgt{\char`\>}
\def\PYZsh{\char`\#}
\def\PYZpc{\char`\%}
\def\PYZdl{\char`\$}
\def\PYZhy{\char`\-}
\def\PYZsq{\char`\'}
\def\PYZdq{\char`\"}
\def\PYZti{\char`\~}
% for compatibility with earlier versions
\def\PYZat{@}
\def\PYZlb{[}
\def\PYZrb{]}
\makeatother


    % Exact colors from NB
    \definecolor{incolor}{rgb}{0.0, 0.0, 0.5}
    \definecolor{outcolor}{rgb}{0.545, 0.0, 0.0}



    
    % Prevent overflowing lines due to hard-to-break entities
    \sloppy 
    % Setup hyperref package
    \hypersetup{
      breaklinks=true,  % so long urls are correctly broken across lines
      colorlinks=true,
      urlcolor=urlcolor,
      linkcolor=linkcolor,
      citecolor=citecolor,
      }
    % Slightly bigger margins than the latex defaults
    
    \geometry{verbose,tmargin=1in,bmargin=1in,lmargin=1in,rmargin=1in}
    
    

    \begin{document}
    
    
    \maketitle
    
    

    
    \section{Statistics from Stock Data}\label{statistics-from-stock-data}

In this lab we will load stock data into a Pandas Dataframe and
calculate some statistics on it. We will be working with stock data from
Google, Apple, and Amazon. All the stock data was downloaded from yahoo
finance in CSV format. In your workspace you should have a file named
GOOG.csv containing the Google stock data, a file named AAPL.csv
containing the Apple stock data, and a file named AMZN.csv containing
the Amazon stock data. (You can see the workspace folder by clicking on
the Jupyter logo in the upper left corner of the workspace.) All the
files contain 7 columns of data:

\textbf{Date Open High Low Close Adj\_Close Volume}

We will start by reading in any of the above CSV files into a DataFrame
and see what the data looks like.

    \begin{Verbatim}[commandchars=\\\{\}]
{\color{incolor}In [{\color{incolor}140}]:} \PY{c+c1}{\PYZsh{} We import pandas into Python}
          \PY{k+kn}{import} \PY{n+nn}{pandas} \PY{k}{as} \PY{n+nn}{pd}
          
          \PY{c+c1}{\PYZsh{} We read in a stock data data file into a data frame and see what it looks like}
          \PY{n}{df} \PY{o}{=} \PY{n}{pd}\PY{o}{.}\PY{n}{read\PYZus{}csv}\PY{p}{(}\PY{l+s+s1}{\PYZsq{}}\PY{l+s+s1}{./GOOG.csv}\PY{l+s+s1}{\PYZsq{}}\PY{p}{)}
          
          \PY{c+c1}{\PYZsh{} We display the first 5 rows of the DataFrame}
          \PY{n}{df}\PY{o}{.}\PY{n}{head}\PY{p}{(}\PY{p}{)}
\end{Verbatim}


\begin{Verbatim}[commandchars=\\\{\}]
{\color{outcolor}Out[{\color{outcolor}140}]:}          Date       Open       High        Low      Close  Adj Close    Volume
          0  2004-08-19  49.676899  51.693783  47.669952  49.845802  49.845802  44994500
          1  2004-08-20  50.178635  54.187561  49.925285  53.805050  53.805050  23005800
          2  2004-08-23  55.017166  56.373344  54.172661  54.346527  54.346527  18393200
          3  2004-08-24  55.260582  55.439419  51.450363  52.096165  52.096165  15361800
          4  2004-08-25  52.140873  53.651051  51.604362  52.657513  52.657513   9257400
\end{Verbatim}
            
    We clearly see that the Dataframe is has automatically labeled the row
indices using integers and has labeled the columns of the DataFrame
using the names of the columns in the CSV files.

\section{To Do}\label{to-do}

You will now load the stock data from Google, Apple, and Amazon into
separte DataFrames. However, for each stock data you will only be
interested in loading the \texttt{Date} and \texttt{Adj\ Close} columns
into the Dataframe. In addtion, you want to use the \texttt{Date} column
as your row index. Finally, you want the DataFrame to recognize the
dates as actual dates (year/month/day) and not as strings. For each
stock, you can accomplish all theses things in just one line of code by
using the appropiate keywords in the \texttt{pd.read\_csv()} function.
Here are a few hints:

\begin{itemize}
\item
  Use the \texttt{index\_col} keyword to indicate which column you want
  to use as an index. For example
  \texttt{index\_col\ =\ {[}\textquotesingle{}Open\textquotesingle{}{]}}
\item
  Set the \texttt{parse\_dates} keyword equal to \texttt{True} to
  convert the Dates into real dates of the form year/month/day
\item
  Use the \texttt{usecols} keyword to select which columns you want to
  load into the DataFrame. For example
  \texttt{usecols\ =\ {[}\textquotesingle{}Open\textquotesingle{},\ \textquotesingle{}High\textquotesingle{}{]}}
\end{itemize}

Fill in the code below:

    \begin{Verbatim}[commandchars=\\\{\}]
{\color{incolor}In [{\color{incolor}141}]:} \PY{c+c1}{\PYZsh{} We load the Google stock data into a DataFrame}
          \PY{n}{google\PYZus{}stock} \PY{o}{=} \PY{n}{pd}\PY{o}{.}\PY{n}{read\PYZus{}csv}\PY{p}{(}\PY{l+s+s1}{\PYZsq{}}\PY{l+s+s1}{./GOOG.csv}\PY{l+s+s1}{\PYZsq{}}\PY{p}{,} \PY{n}{index\PYZus{}col}\PY{o}{=}\PY{p}{[}\PY{l+s+s1}{\PYZsq{}}\PY{l+s+s1}{Date}\PY{l+s+s1}{\PYZsq{}}\PY{p}{]}\PY{p}{,} \PY{n}{parse\PYZus{}dates}\PY{o}{=}\PY{k+kc}{True}\PY{p}{,} \PY{n}{usecols}\PY{o}{=}\PY{p}{[}\PY{l+s+s1}{\PYZsq{}}\PY{l+s+s1}{Date}\PY{l+s+s1}{\PYZsq{}}\PY{p}{,} \PY{l+s+s1}{\PYZsq{}}\PY{l+s+s1}{Adj Close}\PY{l+s+s1}{\PYZsq{}}\PY{p}{]}\PY{p}{)}
          
          \PY{c+c1}{\PYZsh{} We load the Apple stock data into a DataFrame}
          \PY{n}{apple\PYZus{}stock} \PY{o}{=} \PY{n}{pd}\PY{o}{.}\PY{n}{read\PYZus{}csv}\PY{p}{(}\PY{l+s+s1}{\PYZsq{}}\PY{l+s+s1}{./AAPL.csv}\PY{l+s+s1}{\PYZsq{}}\PY{p}{,} \PY{n}{index\PYZus{}col}\PY{o}{=}\PY{p}{[}\PY{l+s+s1}{\PYZsq{}}\PY{l+s+s1}{Date}\PY{l+s+s1}{\PYZsq{}}\PY{p}{]}\PY{p}{,} \PY{n}{parse\PYZus{}dates}\PY{o}{=}\PY{k+kc}{True}\PY{p}{,} \PY{n}{usecols}\PY{o}{=}\PY{p}{[}\PY{l+s+s1}{\PYZsq{}}\PY{l+s+s1}{Date}\PY{l+s+s1}{\PYZsq{}}\PY{p}{,} \PY{l+s+s1}{\PYZsq{}}\PY{l+s+s1}{Adj Close}\PY{l+s+s1}{\PYZsq{}}\PY{p}{]}\PY{p}{)}
          
          \PY{c+c1}{\PYZsh{} We load the Amazon stock data into a DataFrame}
          \PY{n}{amazon\PYZus{}stock} \PY{o}{=} \PY{n}{pd}\PY{o}{.}\PY{n}{read\PYZus{}csv}\PY{p}{(}\PY{l+s+s1}{\PYZsq{}}\PY{l+s+s1}{./AMZN.csv}\PY{l+s+s1}{\PYZsq{}}\PY{p}{,} \PY{n}{index\PYZus{}col}\PY{o}{=}\PY{p}{[}\PY{l+s+s1}{\PYZsq{}}\PY{l+s+s1}{Date}\PY{l+s+s1}{\PYZsq{}}\PY{p}{]}\PY{p}{,} \PY{n}{parse\PYZus{}dates}\PY{o}{=}\PY{k+kc}{True}\PY{p}{,} \PY{n}{usecols}\PY{o}{=}\PY{p}{[}\PY{l+s+s1}{\PYZsq{}}\PY{l+s+s1}{Date}\PY{l+s+s1}{\PYZsq{}}\PY{p}{,} \PY{l+s+s1}{\PYZsq{}}\PY{l+s+s1}{Adj Close}\PY{l+s+s1}{\PYZsq{}}\PY{p}{]}\PY{p}{)}
\end{Verbatim}


    You can check that you have loaded the data correctly by displaying the
head of the DataFrames.

    \begin{Verbatim}[commandchars=\\\{\}]
{\color{incolor}In [{\color{incolor}142}]:} \PY{c+c1}{\PYZsh{} We display the google\PYZus{}stock DataFrame}
          \PY{n}{google\PYZus{}stock}\PY{o}{.}\PY{n}{head}\PY{p}{(}\PY{p}{)}
\end{Verbatim}


\begin{Verbatim}[commandchars=\\\{\}]
{\color{outcolor}Out[{\color{outcolor}142}]:}             Adj Close
          Date                 
          2004-08-19  49.845802
          2004-08-20  53.805050
          2004-08-23  54.346527
          2004-08-24  52.096165
          2004-08-25  52.657513
\end{Verbatim}
            
    \begin{Verbatim}[commandchars=\\\{\}]
{\color{incolor}In [{\color{incolor}143}]:} \PY{n}{amazon\PYZus{}stock}\PY{o}{.}\PY{n}{head}\PY{p}{(}\PY{l+m+mi}{3}\PY{p}{)}
\end{Verbatim}


\begin{Verbatim}[commandchars=\\\{\}]
{\color{outcolor}Out[{\color{outcolor}143}]:}             Adj Close
          Date                 
          2000-01-03    89.3750
          2000-01-04    81.9375
          2000-01-05    69.7500
\end{Verbatim}
            
    \begin{Verbatim}[commandchars=\\\{\}]
{\color{incolor}In [{\color{incolor}144}]:} \PY{n}{apple\PYZus{}stock}\PY{o}{.}\PY{n}{head}\PY{p}{(}\PY{l+m+mi}{3}\PY{p}{)}
\end{Verbatim}


\begin{Verbatim}[commandchars=\\\{\}]
{\color{outcolor}Out[{\color{outcolor}144}]:}             Adj Close
          Date                 
          2000-01-03   3.596616
          2000-01-04   3.293384
          2000-01-05   3.341579
\end{Verbatim}
            
    You will now join the three DataFrames above to create a single new
DataFrame that contains all the \texttt{Adj\ Close} for all the stocks.
Let's start by creating an empty DataFrame that has as row indices
calendar days between \texttt{2000-01-01} and \texttt{2016-12-31}. We
will use the \texttt{pd.date\_range()} function to create the calendar
dates first and then we will create a DataFrame that uses those dates as
row indices:

    \begin{Verbatim}[commandchars=\\\{\}]
{\color{incolor}In [{\color{incolor}145}]:} \PY{c+c1}{\PYZsh{} We create calendar dates between \PYZsq{}2000\PYZhy{}01\PYZhy{}01\PYZsq{} and  \PYZsq{}2016\PYZhy{}12\PYZhy{}31\PYZsq{}}
          \PY{n}{dates} \PY{o}{=} \PY{n}{pd}\PY{o}{.}\PY{n}{date\PYZus{}range}\PY{p}{(}\PY{l+s+s1}{\PYZsq{}}\PY{l+s+s1}{2000\PYZhy{}01\PYZhy{}01}\PY{l+s+s1}{\PYZsq{}}\PY{p}{,} \PY{l+s+s1}{\PYZsq{}}\PY{l+s+s1}{2016\PYZhy{}12\PYZhy{}31}\PY{l+s+s1}{\PYZsq{}}\PY{p}{)}
          
          \PY{c+c1}{\PYZsh{} We create and empty DataFrame that uses the above dates as indices}
          \PY{n}{all\PYZus{}stocks} \PY{o}{=} \PY{n}{pd}\PY{o}{.}\PY{n}{DataFrame}\PY{p}{(}\PY{n}{index} \PY{o}{=} \PY{n}{dates}\PY{p}{)}
\end{Verbatim}


    \section{To Do}\label{to-do}

You will now join the the individual DataFrames, \texttt{google\_stock},
\texttt{apple\_stock}, and \texttt{amazon\_stock}, to the
\texttt{all\_stocks} DataFrame. However, before you do this, it is
necessary that you change the name of the columns in each of the three
dataframes. This is because the column labels in the
\texttt{all\_stocks} dataframe must be unique. Since all the columns in
the individual dataframes have the same name, \texttt{Adj\ Close}, we
must change them to the stock name before joining them. In the space
below change the column label \texttt{Adj\ Close} of each individual
dataframe to the name of the corresponding stock. You can do this by
using the \texttt{pd.DataFrame.rename()} function.

    \begin{Verbatim}[commandchars=\\\{\}]
{\color{incolor}In [{\color{incolor}147}]:} \PY{c+c1}{\PYZsh{} Change the Adj Close column label to Google}
          \PY{n}{google\PYZus{}stock}\PY{o}{.}\PY{n}{rename}\PY{p}{(}\PY{n}{columns}\PY{o}{=}\PY{p}{\PYZob{}}\PY{l+s+s1}{\PYZsq{}}\PY{l+s+s1}{Adj Close}\PY{l+s+s1}{\PYZsq{}}\PY{p}{:} \PY{l+s+s1}{\PYZsq{}}\PY{l+s+s1}{google\PYZus{}stock}\PY{l+s+s1}{\PYZsq{}}\PY{p}{\PYZcb{}}\PY{p}{,} \PY{n}{inplace}\PY{o}{=}\PY{k+kc}{True}\PY{p}{)}
\end{Verbatim}


    \begin{Verbatim}[commandchars=\\\{\}]
{\color{incolor}In [{\color{incolor}148}]:} \PY{c+c1}{\PYZsh{} Change the Adj Close column label to Amazon}
          \PY{n}{amazon\PYZus{}stock}\PY{o}{.}\PY{n}{rename}\PY{p}{(}\PY{n}{columns}\PY{o}{=}\PY{p}{\PYZob{}}\PY{l+s+s1}{\PYZsq{}}\PY{l+s+s1}{Adj Close}\PY{l+s+s1}{\PYZsq{}}\PY{p}{:} \PY{l+s+s1}{\PYZsq{}}\PY{l+s+s1}{amazon\PYZus{}stock}\PY{l+s+s1}{\PYZsq{}}\PY{p}{\PYZcb{}}\PY{p}{,} \PY{n}{inplace}\PY{o}{=}\PY{k+kc}{True}\PY{p}{)}
\end{Verbatim}


    \begin{Verbatim}[commandchars=\\\{\}]
{\color{incolor}In [{\color{incolor}149}]:} \PY{c+c1}{\PYZsh{} Change the Adj Close column label to Apple}
          \PY{n}{apple\PYZus{}stock}\PY{o}{.}\PY{n}{rename}\PY{p}{(}\PY{n}{columns}\PY{o}{=}\PY{p}{\PYZob{}}\PY{l+s+s1}{\PYZsq{}}\PY{l+s+s1}{Adj Close}\PY{l+s+s1}{\PYZsq{}}\PY{p}{:} \PY{l+s+s1}{\PYZsq{}}\PY{l+s+s1}{apple\PYZus{}stock}\PY{l+s+s1}{\PYZsq{}}\PY{p}{\PYZcb{}}\PY{p}{,} \PY{n}{inplace}\PY{o}{=}\PY{k+kc}{True}\PY{p}{)}
\end{Verbatim}


    \begin{Verbatim}[commandchars=\\\{\}]
{\color{incolor}In [{\color{incolor}150}]:} \PY{n}{google\PYZus{}stock}\PY{o}{.}\PY{n}{head}\PY{p}{(}\PY{p}{)}
\end{Verbatim}


\begin{Verbatim}[commandchars=\\\{\}]
{\color{outcolor}Out[{\color{outcolor}150}]:}             google\_stock
          Date                    
          2004-08-19     49.845802
          2004-08-20     53.805050
          2004-08-23     54.346527
          2004-08-24     52.096165
          2004-08-25     52.657513
\end{Verbatim}
            
    You can check that the column labels have been changed correctly by
displaying the datadrames

    \begin{Verbatim}[commandchars=\\\{\}]
{\color{incolor}In [{\color{incolor}151}]:} \PY{c+c1}{\PYZsh{} We display the google\PYZus{}stock DataFrame}
          \PY{n}{amazon\PYZus{}stock}\PY{o}{.}\PY{n}{head}\PY{p}{(}\PY{p}{)}
\end{Verbatim}


\begin{Verbatim}[commandchars=\\\{\}]
{\color{outcolor}Out[{\color{outcolor}151}]:}             amazon\_stock
          Date                    
          2000-01-03       89.3750
          2000-01-04       81.9375
          2000-01-05       69.7500
          2000-01-06       65.5625
          2000-01-07       69.5625
\end{Verbatim}
            
    Now that we have unique column labels, we can join the individual
DataFrames to the \texttt{all\_stocks} DataFrame. For this we will use
the \texttt{dataframe.join()} function. The function
\texttt{dataframe1.join(dataframe2)} joins \texttt{dataframe1} with
\texttt{dataframe2}. We will join each dataframe one by one to the
\texttt{all\_stocks} dataframe. Fill in the code below to join the
dataframes, the first join has been made for you:

    \begin{Verbatim}[commandchars=\\\{\}]
{\color{incolor}In [{\color{incolor}152}]:} \PY{c+c1}{\PYZsh{} We join the Google stock to all\PYZus{}stocks}
          \PY{n}{all\PYZus{}stocks} \PY{o}{=} \PY{n}{all\PYZus{}stocks}\PY{o}{.}\PY{n}{join}\PY{p}{(}\PY{n}{google\PYZus{}stock}\PY{p}{)}
          
          \PY{c+c1}{\PYZsh{} We join the Apple stock to all\PYZus{}stocks}
          \PY{n}{all\PYZus{}stocks} \PY{o}{=} \PY{n}{all\PYZus{}stocks}\PY{o}{.}\PY{n}{join}\PY{p}{(}\PY{n}{apple\PYZus{}stock}\PY{p}{)}
          
          \PY{c+c1}{\PYZsh{} We join the Amazon stock to all\PYZus{}stocks}
          \PY{n}{all\PYZus{}stocks} \PY{o}{=} \PY{n}{all\PYZus{}stocks}\PY{o}{.}\PY{n}{join}\PY{p}{(}\PY{n}{amazon\PYZus{}stock}\PY{p}{)}
\end{Verbatim}


    You can check that the dataframes have been joined correctly by
displaying the \texttt{all\_stocks} dataframe

    \begin{Verbatim}[commandchars=\\\{\}]
{\color{incolor}In [{\color{incolor}153}]:} \PY{c+c1}{\PYZsh{} We display the google\PYZus{}stock DataFrame}
          \PY{n}{all\PYZus{}stocks}\PY{o}{.}\PY{n}{head}\PY{p}{(}\PY{p}{)}
\end{Verbatim}


\begin{Verbatim}[commandchars=\\\{\}]
{\color{outcolor}Out[{\color{outcolor}153}]:}             google\_stock  apple\_stock  amazon\_stock
          2000-01-01           NaN          NaN           NaN
          2000-01-02           NaN          NaN           NaN
          2000-01-03           NaN     3.596616       89.3750
          2000-01-04           NaN     3.293384       81.9375
          2000-01-05           NaN     3.341579       69.7500
\end{Verbatim}
            
    \section{To Do}\label{to-do}

Before we proceed to get some statistics on the stock data, let's first
check that we don't have any \emph{NaN} values. In the space below check
if there are any \emph{NaN} values in the \texttt{all\_stocks}
dataframe. If there are any, remove any rows that have \emph{NaN}
values:

    \begin{Verbatim}[commandchars=\\\{\}]
{\color{incolor}In [{\color{incolor}154}]:} \PY{c+c1}{\PYZsh{} Check if there are any NaN values in the all\PYZus{}stocks dataframe}
          \PY{n}{all\PYZus{}stocks}\PY{o}{.}\PY{n}{isnull}\PY{p}{(}\PY{p}{)}
          
          \PY{c+c1}{\PYZsh{} Remove any rows that contain NaN values}
          \PY{n}{all\PYZus{}stocks}\PY{o}{.}\PY{n}{dropna}\PY{p}{(}\PY{n}{axis}\PY{o}{=}\PY{l+m+mi}{0}\PY{p}{)}
\end{Verbatim}


\begin{Verbatim}[commandchars=\\\{\}]
{\color{outcolor}Out[{\color{outcolor}154}]:}             google\_stock  apple\_stock  amazon\_stock
          2004-08-19     49.845802     1.973460     38.630001
          2004-08-20     53.805050     1.979244     39.509998
          2004-08-23     54.346527     1.997236     39.450001
          2004-08-24     52.096165     2.053144     39.049999
          2004-08-25     52.657513     2.123831     40.299999
          2004-08-26     53.606342     2.227291     40.189999
          2004-08-27     52.732029     2.207371     39.900002
          2004-08-30     50.675404     2.192590     38.310001
          2004-08-31     50.854240     2.216367     38.139999
          2004-09-01     49.801090     2.304405     38.240002
          2004-09-02     50.427021     2.291552     39.180000
          2004-09-03     49.681866     2.263920     38.740002
          2004-09-07     50.461796     2.297979     38.509998
          2004-09-08     50.819469     2.335893     38.009998
          2004-09-09     50.824436     2.294122     38.070000
          2004-09-10     52.324677     2.305047     38.570000
          2004-09-13     53.402668     2.287054     40.009998
          2004-09-14     55.384777     2.280628     42.669998
          2004-09-15     55.638126     2.261993     42.209999
          2004-09-16     56.616764     2.335893     42.570000
          2004-09-17     58.365391     2.386659     42.959999
          2004-09-20     59.294346     2.423287     43.270000
          2004-09-21     58.539257     2.442566     43.290001
          2004-09-22     58.807514     2.372521     41.380001
          2004-09-23     60.019630     2.395013     41.830002
          2004-09-24     59.527828     2.396298     40.939999
          2004-09-27     58.747902     2.411721     39.930000
          2004-09-28     63.020115     2.444494     39.430000
          2004-09-29     65.116478     2.485621     40.840000
          2004-09-30     64.381264     2.490119     40.860001
          {\ldots}                  {\ldots}          {\ldots}           {\ldots}
          2016-11-17    771.229980   108.598877    756.400024
          2016-11-18    760.539978   108.707527    760.159973
          2016-11-21    769.200012   110.357010    780.000000
          2016-11-22    768.270020   110.426147    785.330017
          2016-11-23    760.989990   109.863159    780.119995
          2016-11-25    761.679993   110.416275    780.369995
          2016-11-28    768.239990   110.198975    766.770020
          2016-11-29    770.840027   110.090332    762.520020
          2016-11-30    758.039978   109.161880    750.570007
          2016-12-01    747.919983   108.144531    743.650024
          2016-12-02    750.500000   108.549500    740.340027
          2016-12-05    762.520020   107.769203    759.359985
          2016-12-06    759.109985   108.598877    764.719971
          2016-12-07    771.190002   109.665604    770.419983
          2016-12-08    776.419983   110.742218    767.330017
          2016-12-09    789.289978   112.549728    768.659973
          2016-12-12    789.270020   111.907722    760.119995
          2016-12-13    796.099976   113.774490    774.340027
          2016-12-14    797.070007   113.774490    768.820007
          2016-12-15    797.849976   114.396751    761.000000
          2016-12-16    790.799988   114.544907    757.770020
          2016-12-19    794.200012   115.206673    766.000000
          2016-12-20    796.419983   115.512863    771.219971
          2016-12-21    794.559998   115.621513    770.599976
          2016-12-22    791.260010   114.860977    766.340027
          2016-12-23    789.909973   115.088142    760.590027
          2016-12-27    791.549988   115.819054    771.400024
          2016-12-28    785.049988   115.325203    772.130005
          2016-12-29    782.789978   115.295570    765.150024
          2016-12-30    771.820007   114.396751    749.869995
          
          [3115 rows x 3 columns]
\end{Verbatim}
            
    Now that you have eliminated any \emph{NaN} values we can now calculate
some basic statistics on the stock prices. Fill in the code below

    \begin{Verbatim}[commandchars=\\\{\}]
{\color{incolor}In [{\color{incolor}155}]:} \PY{c+c1}{\PYZsh{} Print the average stock price for each stock}
          \PY{n+nb}{print}\PY{p}{(}\PY{l+s+s1}{\PYZsq{}}\PY{l+s+s1}{mean:}\PY{l+s+se}{\PYZbs{}n}\PY{l+s+s1}{ }\PY{l+s+s1}{\PYZsq{}}\PY{p}{,}\PY{n}{all\PYZus{}stocks}\PY{o}{.}\PY{n}{mean}\PY{p}{(}\PY{n}{axis}\PY{o}{=}\PY{l+m+mi}{0}\PY{p}{)}\PY{p}{)}
          \PY{n+nb}{print}\PY{p}{(}\PY{p}{)}
          
          \PY{c+c1}{\PYZsh{} Print the median stock price for each stock}
          \PY{n+nb}{print}\PY{p}{(}\PY{l+s+s1}{\PYZsq{}}\PY{l+s+s1}{median:}\PY{l+s+se}{\PYZbs{}n}\PY{l+s+s1}{ }\PY{l+s+s1}{\PYZsq{}}\PY{p}{,} \PY{n}{all\PYZus{}stocks}\PY{o}{.}\PY{n}{median}\PY{p}{(}\PY{p}{)}\PY{p}{)}
          \PY{n+nb}{print}\PY{p}{(}\PY{p}{)}
          \PY{c+c1}{\PYZsh{} Print the standard deviation of the stock price for each stock  }
          \PY{n+nb}{print}\PY{p}{(}\PY{l+s+s1}{\PYZsq{}}\PY{l+s+s1}{std:}\PY{l+s+se}{\PYZbs{}n}\PY{l+s+s1}{ }\PY{l+s+s1}{\PYZsq{}}\PY{p}{,} \PY{n}{all\PYZus{}stocks}\PY{o}{.}\PY{n}{std}\PY{p}{(}\PY{p}{)}\PY{p}{)}
          \PY{n+nb}{print}\PY{p}{(}\PY{p}{)}
          \PY{c+c1}{\PYZsh{} Print the correlation between stocks}
          \PY{n+nb}{print}\PY{p}{(}\PY{l+s+s1}{\PYZsq{}}\PY{l+s+s1}{corr:}\PY{l+s+se}{\PYZbs{}n}\PY{l+s+s1}{\PYZsq{}}\PY{p}{,} \PY{n}{all\PYZus{}stocks}\PY{o}{.}\PY{n}{corr}\PY{p}{(}\PY{p}{)}\PY{p}{)}
\end{Verbatim}


    \begin{Verbatim}[commandchars=\\\{\}]
mean:
  google\_stock    347.420229
apple\_stock      35.222976
amazon\_stock    166.095436
dtype: float64

median:
  google\_stock    286.397247
apple\_stock      17.524017
amazon\_stock     76.980003
dtype: float64

std:
  google\_stock    187.671596
apple\_stock      37.945557
amazon\_stock    189.212345
dtype: float64

corr:
               google\_stock  apple\_stock  amazon\_stock
google\_stock      1.000000     0.900242      0.952444
apple\_stock       0.900242     1.000000      0.906296
amazon\_stock      0.952444     0.906296      1.000000

    \end{Verbatim}

    We will now look at how we can compute some rolling statistics, also
known as moving statistics. We can calculate for example the rolling
mean (moving average) of the Google stock price by using the Pandas
\texttt{dataframe.rolling().mean()} method. The
\texttt{dataframe.rolling(N).mean()} calculates the rolling mean over an
\texttt{N}-day window. In other words, we can take a look at the average
stock price every \texttt{N} days using the above method. Fill in the
code below to calculate the average stock price every 150 days for
Google stock

    \begin{Verbatim}[commandchars=\\\{\}]
{\color{incolor}In [{\color{incolor}179}]:} \PY{c+c1}{\PYZsh{} We compute the rolling mean using a 3\PYZhy{}Day window for  stock}
          \PY{n}{rollingMean} \PY{o}{=} \PY{n}{all\PYZus{}stocks}\PY{p}{[}\PY{l+s+s1}{\PYZsq{}}\PY{l+s+s1}{amazon\PYZus{}stock}\PY{l+s+s1}{\PYZsq{}}\PY{p}{]}\PY{o}{.}\PY{n}{rolling}\PY{p}{(}\PY{l+m+mi}{3}\PY{p}{)}\PY{o}{.}\PY{n}{mean}\PY{p}{(}\PY{p}{)}
          \PY{n+nb}{print}\PY{p}{(}\PY{n}{rollingMean}\PY{p}{)}
\end{Verbatim}


    \begin{Verbatim}[commandchars=\\\{\}]
2000-01-01           NaN
2000-01-02           NaN
2000-01-03           NaN
2000-01-04           NaN
2000-01-05     80.354167
2000-01-06     72.416667
2000-01-07     68.291667
2000-01-08           NaN
2000-01-09           NaN
2000-01-10           NaN
2000-01-11           NaN
2000-01-12     66.500000
2000-01-13     65.416667
2000-01-14     64.583333
2000-01-15           NaN
2000-01-16           NaN
2000-01-17           NaN
2000-01-18           NaN
2000-01-19           NaN
2000-01-20     65.229167
2000-01-21     64.541667
2000-01-22           NaN
2000-01-23           NaN
2000-01-24           NaN
2000-01-25           NaN
2000-01-26     68.062500
2000-01-27     67.000000
2000-01-28     64.479167
2000-01-29           NaN
2000-01-30           NaN
                 {\ldots}    
2016-12-02    744.853353
2016-12-03           NaN
2016-12-04           NaN
2016-12-05           NaN
2016-12-06           NaN
2016-12-07    764.833313
2016-12-08    767.489990
2016-12-09    768.803324
2016-12-10           NaN
2016-12-11           NaN
2016-12-12           NaN
2016-12-13           NaN
2016-12-14    767.760010
2016-12-15    768.053345
2016-12-16    762.530009
2016-12-17           NaN
2016-12-18           NaN
2016-12-19           NaN
2016-12-20           NaN
2016-12-21    769.273316
2016-12-22    769.386658
2016-12-23    765.843343
2016-12-24           NaN
2016-12-25           NaN
2016-12-26           NaN
2016-12-27           NaN
2016-12-28           NaN
2016-12-29    769.560018
2016-12-30    762.383341
2016-12-31           NaN
Freq: D, Name: amazon\_stock, Length: 6210, dtype: float64

    \end{Verbatim}

    We can also visualize the rolling mean by plotting the data in our
dataframe. In the following lessons you will learn how to use
\textbf{Matplotlib} to visualize data. For now I will just import
matplotlib and plot the Google stock data on top of the rolling mean.
You can play around by changing the rolling mean window and see how the
plot changes.

    \begin{Verbatim}[commandchars=\\\{\}]
{\color{incolor}In [{\color{incolor}180}]:} \PY{c+c1}{\PYZsh{} this allows plots to be rendered in the notebook}
          \PY{o}{\PYZpc{}}\PY{k}{matplotlib} inline 
          
          \PY{c+c1}{\PYZsh{} We import matplotlib into Python}
          \PY{k+kn}{import} \PY{n+nn}{matplotlib}\PY{n+nn}{.}\PY{n+nn}{pyplot} \PY{k}{as} \PY{n+nn}{plt}
          
          
          \PY{c+c1}{\PYZsh{} We plot the Amazon stock data}
          \PY{n}{plt}\PY{o}{.}\PY{n}{plot}\PY{p}{(}\PY{n}{all\PYZus{}stocks}\PY{p}{[}\PY{l+s+s1}{\PYZsq{}}\PY{l+s+s1}{amazon\PYZus{}stock}\PY{l+s+s1}{\PYZsq{}}\PY{p}{]}\PY{p}{)}
          
          \PY{c+c1}{\PYZsh{} We plot the rolling mean ontop of our Amazon stock data}
          \PY{n}{plt}\PY{o}{.}\PY{n}{plot}\PY{p}{(}\PY{n}{rollingMean}\PY{p}{)}
          \PY{n}{plt}\PY{o}{.}\PY{n}{legend}\PY{p}{(}\PY{p}{[}\PY{l+s+s1}{\PYZsq{}}\PY{l+s+s1}{Amazon Stock Price}\PY{l+s+s1}{\PYZsq{}}\PY{p}{,} \PY{l+s+s1}{\PYZsq{}}\PY{l+s+s1}{Rolling Mean}\PY{l+s+s1}{\PYZsq{}}\PY{p}{]}\PY{p}{)}
          \PY{n}{plt}\PY{o}{.}\PY{n}{show}\PY{p}{(}\PY{p}{)}
\end{Verbatim}


    \begin{center}
    \adjustimage{max size={0.9\linewidth}{0.9\paperheight}}{output_28_0.png}
    \end{center}
    { \hspace*{\fill} \\}
    

    % Add a bibliography block to the postdoc
    
    
    
    \end{document}
